\section{Lens distortion and point discretization}
\label{sec:model-lens-distortion}
In Section \ref{subsec:lenses} we have introduced lenses, highlighting their advantages over the pinhole. We also introduced the differences between the thin lens model and the thick lens one, observing how, in both cases, curvature radii could introduce aberrations on the acquired image.

As we said in Section \ref{sec:teo-calibration}, all the calibration algorithms try to solve this problem, but most of them, like Tsai and Zhang, are limited to considering radial distortions. In fact it can be shown that tangential contributions are typically negligible, while radial ones increase when focal length decreases. Fortunately, in literature there are some studies about lenses and thin prism distortions, such as \cite{brown},\cite{DBLP:journals/corr/cs-CV-0308003} and \cite{Heikkila}, which offer some solutions to extend the analysis beyond this limit. \\

As a first thing, we focused on \cite{TsaiTvLenses}. Accordingly with it, the commonly used polynomial for radial distortion model, is given by the series
  \begin{equation}
    \label{eq:radial-tot}
    r_d = r + \delta_r = r + \sum_{j=1}^\infty k_jr^{(2j)}
  \end{equation}
where $r$ is the lens radius and $k_j$ is the radial coefficient of degree $j$. As we will discuss later, the used calibration algorithm are limited to the second order, so Equation \ref{eq:radial-tot} can be reduced as:
  \begin{equation}
    \label{eq:radial-2}
    r_d = r + k_1r^2 + k_2r^4
  \end{equation}

Applying Equation \ref{eq:radial-2} to distorted points, we found a simple relation, similar to that in \cite{TsaiTvLenses}:
  \begin{equation}
    \label{eq:dist-coords}
    x_{p_i} = x_{p_i}^{d} \left( 1 + k_1r^2 \right) \qquad
    y_{p_i} = y_{p_i}^{d} \left( 1 + k_1r^2 \right)
  \end{equation}
where $\left( x_{p_i}^{d}, y_{p_i}^{d} \right)$ are the distorted coordinates in the parallel to sensor image plane. In this way, we can determine the projection of a 3D point in a plane parallel to the sensor and distortion free. Note that, given a specific point, it does not make sense to consider the whole radius $r$ of the lens, so it is preferable to consider its distance from the point to the optical center (that ideally is locate in the center of the lens). \\
As we have done in previous sections, the error is propagated as follows
  \begin{equation}
    \label{eq:sigma-dist}
    \begin{matrix}
      \sigma_{x_{p_i}} = \sqrt{
        \left( \frac{\partial x_{p_i}}{\partial x_{p_i}^d} \right)^2 \sigma_{x_{p_i}^d}^2
        + \left( \frac{\partial x_{p_i}}{\partial k_1} \right)^2 \sigma_{k_1}^2
      }
      \\~\\
      \sigma_{y_{p_i}} = \sqrt{
        \left( \frac{\partial y_{p_i}}{\partial y_{p_i}^d} \right)^2 \sigma_{y_{p_i}^d}^2
        + \left( \frac{\partial y_{p_i}}{\partial k_1} \right)^2 \sigma_{k_1}^2
      } 
    \end{matrix}
  \end{equation}
Note that this transformation is the same for both $x$ and $y$, thanks to lens radial distortion symmetry. \\

At this point we have noticed that it is very easy to spread the solution to the tangential distortions. Accordingly with \cite{Heikkila} we observed that Equations \ref{eq:radial-2} could be extended easily:
  \begin{equation*}
    \mathcal{F}\left( r, \bar{k}, \bar{p} \right) = 
    \begin{bmatrix}
      x_{p_i}\left( \sum_{j=1}^\infty k_jr^{2j} \right) + \left( 2 p_1 x_{p_i} y_{p_i} + p_2 \left( r^2 + 2 x_{p_i}^2  \right) \right) \left( 1 + p_3 r^2 + \ldots  \right)
      \\
      y_{p_i}\left( \sum_{j=1}^\infty k_jr^{2j} \right) + \left(  p_1 \left( r^2 + 2 y_{p_i}^2  \right) + 2 p_2 x_{p_i} y_{p_i} \right) \left( 1 + p_3 r^2 + \ldots \right)
    \end{bmatrix}
  \end{equation*}
As we can see, to consider tangential distortion some additive factors are needed. The analysis for error propagation is the same as that for the Equation \ref{eq:sigma-dist}, but we have to pay attention to include the partial derivatives of the tangential coefficients. \\
To complete the analysis, we have done some test using nominal lens parameters given by the manufacturer, and we observed that the error improvement was negligible. \\

In the last two cases we could ignore the components depending by distortion coefficients, that we can consider correct thanks to calibration processes. \\

Looking around, we have found another model that we considered of interest. Accordingly with \cite{DBLP:journals/corr/cs-CV-0308003}, radial distortions could be formulated as a rational distribution like:
  \begin{equation*}
    \mathcal{F}\left( r, \bar{k}, \bar{p} \right) = \frac{1 + k_1r + k_2r^2}{1 + k_3r + k_4r^2 + k_5r^3}
  \end{equation*}
It is easy to derive many other formulas from the general one, and the most interesting are shown in Table \ref{tab:dist-funcs}.
  \begin{table}[t!]
  \centering
  \begin{tabular}{r|l}
  \multicolumn{1}{c|}{\textbf{\#}} & \multicolumn{1}{c}{\textbf{$\mathcal{F}\left( r, \bar{k} \right)$}} \\
    \hline
    1	& $1 + k_1 r$ \\
    2	& $1 + k_1 r^2$ \\
    3	& $1 + k_1 r + k_2 r^2$ \\
    4	& $1 + k_1 r^2 + k_2 r^4$ \\
    5	& $1 / (1 + k_1 r) $ \\
    6	& $1 / (1 + k_1 r^2)$ \\
    7	& $(1 + k_1 r) / (1 + k_2 r^2)$ \\
    8	& $1 / (1 + k_1 r + k_2 r^2)$ \\
    9	& $(1 + k_1 r) / (1 + k_2 r + k_3 r^2)$ \\
    10	& $(1 + k_1 r^2) / (1 + k_2 r + k_3 r^2)$ \\
    \hline
  \end{tabular}
  
  \caption{Polynomial and rational distortion functions}
  \label{tab:dist-funcs}
\end{table}
All these functions enjoy some properties,. in fact they are:
  \begin{itemize}
    \item radially symmetric around the center of distortion;
    \item expressed in terms of the radius $r$ only;
    \item continuous and $r_d = 0$ iff $r = 0$;
    \item the approximation of $x_d$ is an odd function of $x$.
  \end{itemize}
The above three properties act as the criteria to be good candidates as radial distribution functions. Despite that, we can see that Equation \#4 is very closed to the Equation \ref{eq:radial-2}. Furthermore, the calibration algorithms we have used are limited to second degree, and in this conditions many of these equations can be traced back to \#4. At the end, it can be shown that the performances of the remaining functions are comparable with \#4. \\
For all these reasons, we focused only on Equations \ref{eq:dist-coords}, but as we have shown in this section, it is easy to extends it to tangential distortion or to increase its degree.

\bigskip
Once we have determined the distorted point in the image plane, the last step is the point discretization. In this phase we are interested in projecting the undistorted point in the distorted sensor plane. Accordingly with \cite{TsaiTvLenses}, this projection can be performed by centering the coordinate reference system in the optical center estimated during calibration processes, and by normalizing the point value in the sensor pixel range. At the end we can write:
  \begin{equation*}
    x_{p_i}^d = d_x'(x_{c_i} - c_x) \qquad \qquad y_{p_i}^d = d_y'(y_{c_i} - c_y)
    % \label{eq:discrete-coords}
  \end{equation*} \\
where $(c_x, c_y)$ is the optical center, and $(d_x', d_y')$ the normalized pixel center to center distance, along $x$ and $y$ axis, respectively. Consistent with what we have done so far, we can conclude with error propagation:
  \begin{equation}
    \begin{matrix}
    %
    \sigma_{x_{p_i}^d} = \sqrt{
      \left( \frac{\partial x_{p_i}^d}{\partial x_{c_i}} \right)^2 \sigma_{x_{c_i}}^2
      + \left( \frac{\partial x_{p_i}^d}{\partial c_x} \right)^2 \sigma_{c_x}^2
    }
  %\end{equation*}
  \\
  %\begin{equation*}
    \sigma_{y_{p_i}^d} = \sqrt{
      \left( \frac{\partial y_{p_i}^d}{\partial y_{c_i}} \right)^2 \sigma_{y_{c_i}}^2
      + \left( \frac{\partial y_{p_i}^d}{\partial c_y} \right)^2 \sigma_{c_y}^2
    }
    %
    \end{matrix}
    \label{eq:model:err:disc}
  \end{equation} \\

\bigskip
Before concluding this section, we want to emphasize a possible source of ambiguity in understanding this model. The explanation started from the end of the chain and follows until the beginning. This choice was taken to better identify, for each step, the elements to analyse, and from which each relation depends. In this way the error determination was easy. However, we have to notice that we want to determine the error committed evaluating the point in the world, starting from pixel detection in the image, then formulas have to interpret in this sense, from pixel to world.
