\begin{table}[b!]
  \begin{tabular}{|c|p{12cm}|}
  \hline
  \multicolumn{1}{|c|}{\textbf{Class}} & \multicolumn{1}{c|}{\textbf{Description}} \\
  \hline
  
  1  &
  \acs{LASER}s that belong to this class are always safe for the human health. Generally the emitted power is $P < 0.04 mW$, but in this class there are also that \acs{LASER}s that prevents the direct interaction of the operator (such as laser printers). No protection is required. \\
  \hline
  
  1M &
  This class differs from the previous one because these are dangerous \acs{LASER}s, if used in pair with optical instruments. Typically, their works in the range $\left[302.5, 4000\right] nm$. \\
  \hline
  
  2  &
  These are low-power \acs{LASER}s $\left( P < 1 mW\right)$ that emit in the visible spectrum $\left( 400, 700 \right) nm$. This type of \acs{LASER}s are not inherently safe, but the eyelid is enough to protect eyes from incidental reflections. It is important not to look them directly. \\
  \hline
  
  2M &
  Like the class 1M, \acs{LASER}s belonging to this class refer to class 2 too, but are dangerous if used in pair with optical instruments. \\
  \hline
  
  3R &
  This is the first class of \acs{LASER}s that are really dangerous for humans. Here we find \acs{LASER}s that emit in the visible spectrum $\left(302.5, 106\right) nm$. They can injure eyes even if viewed indirectly. \\
  \hline
  
  3B &
  Like the class 3R, we can find \acs{LASER}s dangerous both for eyes and skin. They emit beams with power under $500 mW$. \\
  \hline
  
  4  &
  Like the class 3B, \acs{LASER}s that belong to this class are dangerous for eyes and skin, even in diffuse radiation, and they can cause fires. Another source of danger are their very high working voltage and current. We have to be very careful when we use this class. \\
  \hline
  \end{tabular}
    
  \caption{Classes of \acs{LASER}s and their regulations} 
  \label{tab:laser-classes}
\end{table}