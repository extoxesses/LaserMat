\begin{abstract}
In this work, we want to study the problem of prototyping and requirements analysis for laser triangulation-based machinery, in the field of railway wheels measurement systems. To reach this goal, we will focus on developing a complete mathematical model, which will consider all the components of interest. In particular, we will center on the improvements on the system, given by the use of precise laser spots detector in the images. Finally, we will study the error propagation on complex measures, such as the diameter.
\end{abstract}

\clearpage
\begin{center}
  \textit{This page is intentionally left blank}
\end{center}
\clearpage

\selectlanguage{italian}
\begin{abstract}
In questa tesi analizziamo il problema della prototipazione e studio dei requisiti per un sistema a triangolazione laser, con particolare attenzione ai sistemi di misura di ruote ferroviarie. Per raggiungere questo obiettivo, ci concentreremo sullo sviluppo di un modello matematico completo, che permetta di considerare tutti gli elementi di interesse. In particolare ci focalizzaremo sull'estrazione precisa dello spot laser da un'immagine, e sui loro effetti sulle prestazioni finali del sistema. Infine, analizzeremo come gli errori di misura si propagano a misure complesse, come quella del diametro.
\end{abstract}

\clearpage
\begin{center}
  \textit{This page is intentionally left blank}
\end{center}
\clearpage

\selectlanguage{english}
