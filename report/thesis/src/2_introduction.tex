\chapter{Introduction}
\pagenumbering{arabic}

Over the last decades, the exponential growth of robots and automate measuring and handling systems has pushed heavily towards the development of computer vision systems, for the 3D reconstruction of the world: when these systems are used to make very precise measures, it is fundamental to estimate how much performing a system is. Today, there aren't complete mathematical models, for laser triangulation based systems, that allow to analyse how a negligible error, made locating the spot laser in the image, influences the correct estimation of the same in the world, and furthermore, how this error is propagated while we estimate the measures of interest. Regarding complex systems, such as railway \textit{wheel profile measurements systems} (\acs{WPMS}), we have also to consider some derived measures (e.g. the diameter of the wheel), heavily affected by the errors above. If we will be able to control all the sources of noise during the design phase of a new product, we will be able to improve the performance of our measurement systems. From the point of view of a large company, this results in a saving of resources, and a reduction in the time required for the product to propose to the market, with an increase in earnings for the company itself. In addition, a model like this, would allow us to know the lower performance limits of our systems.

In this work, we will study all the details that characterize a laser triangulation system, in particular we will focus on sub-pixel laser detectors, and on their improvements on final measures. Then, we will develop and validate a mathematical model, useful during the design of a new product. In the end, we will focus on the problem of optimizing the accuracy of the diameter measurement.

The study was done on behalf of MER MEC S.p.A.\footnote{\url{http://www.mermecgroup.com/}}, which holds the ownership of data used in experiments.

% --- Structure of the thesis
\section{Thesis structure}
In Chapter \ref{ch:technology}, we will review the literature about the components of interest for the systems under analysis, and for each components we will analyse the properties useful for our project.

In Chapter \ref{ch:sys_cmp}, we will introduce the theory behind the wheel profile in rail systems, discussing how this system are made. Furthermore, we will present some different mathematical models, used in products available on the market.

In Chapter \ref{ch::model}, we will present our work, discussing a proposed mathematical model that avers to take into account all the aspect of interest in laser triangulation based systems.

In Chapter \ref{ch:experimets}, we will discuss the results reached with the model above, and we will try to validate it. 

In Chapter \ref{ch::diameter}, we complete the model, introducing the problem of the evaluation of the diameter of the wheel. Furthermore, we will discuss some results reached with it.

Finally, in Chapter \ref{ch:conclusions}, we will report our conclusions.
