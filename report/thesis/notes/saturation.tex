\newpage

\section*{Modello Gaussiano affetto da rumore\\Analisi matematica dell'errore di approssimazione}

Consideriamo un segnale \textit{Gaussiano}
\begin{equation}
f(n) = e^{-\frac{(n-\delta)^2}{2\sigma^2}}+\frac{\beta}{A}
\end{equation}
con
\begin{itemize}
  \item $n$ posizione del pixel rispetto a quello centrale (centrato per comodità di notazione su $n=0$)
  \item $\delta$ indica l'offset del picco rispetto al pixel $n=0$
  \item $\sigma$ è l'apertura del laser
  \item $\beta$ è l'ampiezza del rumore
  \item $A$ è l'apertura del rumore
\end{itemize}
Inoltre, è possibile considerare con un certo errore di approssimazione (vedi studio sul resto di Lagrange per l'errore di approssimazione) l'approssimazione di Taylor del primo ordine
\begin{equation*}
f(n) \approx \left( 1+\frac{n\delta}{\sigma^2} \right) \cdot e^{-\frac{n^2}{2\sigma^2}}
\end{equation*}

%%%%%%%%%%%%%%%%%%%%%%%%%%%%%%%%%%%%%%%%%%%%%%%%%%%%%%%%%%%%%%%%%%%%%%%%%%%%%%%%%
\subsection*{CoM5}
Il \textit{modello del centro di massa} si definisce come
\begin{equation*}
CoM_5 = \frac{-2a-b+d+2e}{a+b+c+d+e}
\end{equation*}
con
\begin{equation*}
  a=f(-2) \ \ \ b=f(-1) \ \ \ c=f(0) \ \ \ d=f(1) \ \ \ e=f(2)
\end{equation*}
Sviluppando l'equazione avremo:
\begin{equation}
  CoM_5 \approx \frac{2\delta}{\sigma^2} \cdot \frac{
    e^{-\frac{1}{2\sigma^2}}+4e^{-\frac{4}{2\sigma^2}}
  }{
    1+2e^{-\frac{1}{2\sigma^2}}+2e^{-\frac{4}{2\sigma^2}} + \frac{5\beta}{A}
  }
\end{equation}
Quello che si può vedere è che si ottiene un risultato più rumoroso a causa dell'addendo a denominatore. Tuttavia questa approssimazione non credo sia corretta a causa del fatto che stiamo conservando l'informazione sulla Gaussiana di base, cosa che invece non è vera, in quanto perché il modello sia corretto dovrebbe avere un magnitudo molto inferiore rispetto a quello del rumore (per avere saturazione). \\

Per ovviare a questo problema, io utilizzerei una funzione a gradino per il rumore, del tipo
\begin{equation*}
  \mathcal{N} = \left\{
    \begin{matrix}
      \frac{\beta}{A} \ per \ |n| < \frac{A}{2}\\ 
      0 \ altrimenti
    \end{matrix}
  \right.
\end{equation*}
In questo modo è possibile introdurre una funzione che in parte conserva l'informazione sulla distribuzione gaussiana, e dall'altra permette avere una saturazione. Applicando tale segnale si ha che:
\begin{equation}
  CoM_5 \approx
  \frac{
    -2e^{-\frac{(\delta-2)}{2\sigma^2}} - \frac{\beta}{A} + \frac{\beta}{A} + 2e^{-\frac{(\delta-2)}{2\sigma^2}}
  }{
    e^{-\frac{(\delta-2)}{2\sigma^2}} + \frac{3\beta}{A} + e^{-\frac{(\delta+2)}{2\sigma^2}}
  } \approx
  \frac{8\delta}{\sigma^2} \cdot \frac{
    e^{-\frac{4}{2\sigma^2}}
  }{
    2e^{-\frac{4}{2\sigma^2}} + \frac{3\beta}{A}
  }
\end{equation}
In questo modo posso assumere che la saturazione sia dovuta ad un rumore rettangolare che si somma alla funzione originale, della quale conservo solo le code. Posti
\begin{itemize}
  \item $\sigma^2 = 1$
  \item $\beta = 1$ (Visto che finora non anche la gaussiana è stata considerata con gain unitario, stesso faccio per la uniforme)
  \item $A = 2\sigma$ (anche se in realtà dovrebbe bastare considerarlo come l'apertura di un pixel $[0,1]$ come nel paper)
\end{itemize}
l'errore per il $CoM_5$ sarà:
\begin{equation}
  |\delta - \hat{\delta}| = 2.67 \cdot |\delta|
\end{equation}

\noindent
\textbf{NOTA:} fuori da queste ipotesi non sono riuscito a sviluppare un modello che tenesse in considerazione il $\delta$ inteso come variazione della posizione del centro rispetto al pixel. In questi casi qualsiasi elaborazione mi porta ad avere un risultato valutabile in termini quantitativi solo per via analitica e non matematica.

%%%%%%%%%%%%%%%%%%%%%%%%%%%%%%%%%%%%%%%%%%%%%%%%%%%%%%%%%%%%%%%%%%%%%%%%%%%%%%%%%

\subsection*{Blais and Rioux (BR4)}
Secondo quanto riportato dal paper, possiamo modellare le seguenti funzioni
\begin{equation*}
  g_4(n) = f(n-2) + f(n-1) - f(n+1) - f(n+2)
\end{equation*}
cui segue la funzione per il calcolo di $\delta$
\begin{equation*}
  \delta = \frac{g_4(n)}{g_4(n) - g_4(n+1)}
\end{equation*}
Facendo i dovuti calcoli si ha
\begin{equation}
\delta \approx \frac{4\delta}{\sigma^2} \cdot \frac{
    e^{-\frac{4}{2\sigma^2}}
  }{
    \frac{2\beta}{A} + 
    \left(\frac{2\delta}{\sigma^2}-1\right)e^{-\frac{4}{2\sigma^2}} -
    \left(1+\frac{3\delta}{\sigma^2}\right)e^{-\frac{9}{2\sigma^2}}
  }
\end{equation}
che se vogliamo risolvere con l'approssimazione fantasiosa del passo precedente, viene
\begin{equation}
\delta \approx \frac{4\delta}{\sigma^2} \cdot \frac{
    e^{-\frac{4}{2\sigma^2}}
  }{
    \frac{2\beta}{A} -
    e^{-\frac{4}{2\sigma^2}} -
    e^{-\frac{9}{2\sigma^2}}
  }
\end{equation}

\noindent
Nelle stesse ipotesi del caso precedente, possiamo calcolare l'errore
\begin{equation}
|\delta - \hat{\delta}| = 0,29 \cdot |\delta|
\end{equation}